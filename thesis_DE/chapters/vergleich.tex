\chapter{Gegenüberstellung der Methoden}
\label{cha:Vergleich}

In Kapitel ~\ref{cha:Ausgabe} wurden die verschiedenen Eingabemethoden näher erläutert. Dieses Kapitel beschreibt, welche Vor- und Nachteile, gesonderte Rahmenbedingungen und welche Einsatzgebiete es zu den einzelnen Methoden gibt. Da der Fokus dieser Arbeit auf der Steuerung von Systemen liegt, wird in diesem Kapitel auf die verschiedenen Ausgabemethoden nicht eingegangen.

\section{Fördernde und hemmende Faktoren}
%
Jede einzelne Eingabemethode weist gewisse Pro- und Contraseiten auf. Für einen Vergleich wurden verschiedene Kriterien definiert. Tab.~\ref{tab:matrixObj} zeigt die objektiven Kriterien, Tab.~\ref{tab:matrixSubj} stellt die subjektiven Kriterien dar.
\newline \newline
Der erste Vergleichsfaktor, der in Tab.~\ref{tab:matrixObj} dargestellt ist, bezieht sich auf die Kosten. Bei der Sprachsteuerung gibt es verschiedene Sprachassistenten wie \zB Siri%
\footnote{https://www.apple.com/ios/siri/}
%
 oder Cortana%
\footnote{https://support.microsoft.com/en-us/help/17214}
%
 , mit denen ein Computer zum Teil mit Hilfe der Sprache gesteuert werden kann. Die Programme ansich sind kostenlos, wenn der Benutzer über das entsprechende Gerät verfügt (iPhone etc. oder Computer mit Windows10).
Wie in Abschnitt \label{cha:Augensteuerung} erklärt, werden für die Augensteuerung sowohl ein Eye-Tracking-System, als auch eine Software benötigt. Exklusive der Kosten für die Software belaufen sich die in Abb.~\ref{fig:Tobii}(a) dargestellten Tobii Pro Glasses 2 auf rund 13.500€ \cite{TobiiCosts}. Allerdings kann auch eine günstigere Variante implementiert werden. Hierbei benötigt man zwei Webcams (ca. 100€) und eine Open Source Software für die Blickerkennung und Steuerung. Kinn- und Mundsteuerungen kosten zwischen 400€ und 2000€ \cite{SENSORY} \cite{INTEGRA}. Das Myo Armband als Referenz für die Muskelsteuerung beläuft sich auf ca. 250€, wobei komplexere Systeme wesentlich teurer sind \cite{myoBand}. Das in Abb.~\ref{fig:epoc} dargestellte EPOC+, das dem Benutzer ermöglicht mit einem System mit Hilfe von Gehirnaktivität zu interagieren, kostet rund 720€.
\newline \newline \newline \newline \newline
Die meisten System weisen keine räumliche Einschränkung auf und sie sind daher nicht an ein gewisses Setting gebunden. Bei der Benutzung einer Augensteuerung ist allerdings darauf zu achten, dass keine störenden Lichtquellen die Kamera bzw. die Interaktion behindern. Des weiteren ist ein Benutzer bei der Verwendung von Systemen, die durch die Gehirn- oder Muskelaktivität gesteuert werden, oft an die notwendigen Geräte und somit an gewisse Räumlichkeiten gebunden.
\newline \newline
Die Größe und das Gewicht der Systeme richten sich immer nach den Interaktionsgeräten der jeweiligen Eingabemethode. Für die Augensteuerung können beispielsweise die Geräte, die in Abb.~\ref{fig:Tobii} dargestellt sind, verwendet werden. Hier bewegt sich die Spanne von 312g für Tobii Pro Glasses 2 bis zu einem Gesamtgewicht von 8.9kg für die Eye-Tracker-Einheit und den Monitor des Tobii Pro Spectrum. Die Elektroden an sich, die die Basis für die Muskelsteuerung und die Steuerung mit Hilfe von Gehirnaktiviät darstellen sind an sich sehr klein und haben kaum ein Gewicht. Das Myo Armband, das hier als Beispiel angeführt wird, ist nur 11,9 x 7,4 x 10,4 cm groß und 254g schwer. Während das EPOC+ des Unternehmens Emotiv mit 23 x 38 x 38 cm und 540g mehr als doppelt so groß und doppelt so schwer ist.
\newline \newline
Wie zu Beginn schon erwähnt, werden in Tab.~\ref{tab:matrixSubj} subjektive Faktoren als Vergleich der einzelnen Eingabemethoden verwendet.
\newline \newline
Für die Benutzung jeder der verschiedenen Eingabemethoden gelten die selben Grundvoraussetzungen. Der Benutzer muss eine eigenständige Kontrolle über das jeweilige Eingabemittel (Sprache, Muskelanspannung, Augenbewegungen etc.) besitzen, damit eine Interaktion möglich ist. Zusätzlich gibt es Faktoren, die von Methode zu Methode unterschiedlich sind. So ist bei der Sprachsteuerung eine laute und deutliche Sprache ein fördernder Faktor, hingegen erweisen sich verstaubte oder spiegelnde Brillengläser bei der Augensteuerung als ein hemmender Faktor.
\newline \newline
Die Dauer, wie lange ein Benutzer mit einem Computer interagieren kann, ist von System zu System sehr unterschiedlich. Da die Sprachsteuerung und die Gestensteuerung nur eine geringe kognitive und körperliche Anstrengung erfordern, gibt es bei der Benutzungsdauer kaum Einschränkungen. Im Gegensatz dazu findet man bei der Augensteuerung eine hohe körperliche Anstrengung und sollte daher nicht zu lange eingesetzt werden, damit die Augen langfristig nicht geschädigt werden. Ähnlich wie bei der gewöhnlichen Benutzung eines Computers sollte alle 20 Minuten eine Pause eingelegt werden \cite{20Methode}. Das Benutzen einer Steuerung, die durch die Muskeln oder durch die Gehirnaktivität funktioniert, ist mit hoher kognitiven Anstrengung verbunden. Zusätzlich erfordert das Anspannen und Entspannen der Muskelpartien eine hohe körperliche Anstrengung. Aus diesen Gründen ist die Interaktionsdauer der beiden Methoden sehr beschränkt. Daher muss während der Interaktion regelmäßig überprüft werden, ob zum gegebenen Zeitpunkt eine Überanstrengung erfolgt. Ist dies der Fall, sollte eine Pause eingelegt werden, um den beanspruchten Partien eine Erholungsphase zu ermöglichen.
\newline \newline \newline \newline
Wie lange mit einem System interagiert werden kann, hängt unter anderem von der kognitiven und körperlichen Anstrengung ab. Elisa Mira Holz u.a \cite{holz2013brain} entwickelten ein BCI-Prototypen, mit dem es möglich ist mit Hilfe von Gehirnaktivität mit einem Computerspiel zu interagieren. Nach der Interaktion mit dem Prototypen wurde eine Evaluation zur Einschätzung der subjektive Arbeitsbelastung nach dem NASA-TLX-Fragebogen durchgeführt. Festgestellt wurde, dass nur eine geringe körperliche, jedoch aber eine hohe kognitive Anstrengung besteht. Alle anderen Eingabemethoden wurden anhand der Ergebnisse dieser Studie abgeleitet. Die Steuerung durch Muskelaktivität erfordert eine hohe körperliche Anstrengung, da das Muskelan- und entspannen eine hohe Beanspruchung darstellt. Auch die kognitive Anstrengung ist hier sehr hoch, weil sich der Benutzer zum einen sehr konzentrieren muss, um die Muskeln richtig einsetzten zu können, zum anderen ist diese Interaktionsart hinsichtlich der Bedienbarkeit gewöhnungsbedürftig. Im Vergleich zur kognitiven Anstrengung bei der Steuerung mit Hilfe von Gehirnsteuerung, gibt es bei den restlichen Eingabemethoden nur eine geringe Belastung. Da die Augen ein sehr empfindliches Sinnesorgan ist, und die Interaktionsdauer mit einem System beschränkt ist, wird hier eine hohe körperliche Anstrengung angeführt.
\newline \newline
Weil das Sprechen eine sehr natürliche Form der Kommunikation ist, ist auch die Interaktion mit einem Gerät intuitiv. Ähnlich kann über die Gestensteuerung gesagt werden, da das Bewegen des Joysticks in die einzelnen Richtungen sehr einfach und intuitiv ist. Gewöhnungsbedürftiger für den Benutzer ist die Augensteuerung, da er sich wesentlich mehr konzentrieren und stets den Bildschirm fokussieren muss. Auch die Muskelsteuerung und die Steuerung mit Hilfe von Gehirnaktivität sind gewöhnungsbedürftig, da diese Interaktionsformen keine vertraute Art der Kommunikation darstellen und mit hoher kognitiven Anstrengung verbunden sind.
\newline \newline
Bevor ein Benutzer mit einem System interagieren kann, muss ein Gerät oft individuell angepasst (kalibriert) werden. Dies ist bei allen beschriebenen Eingabemethoden, ausschließlich der Gestensteuerung, der Fall. Bei der Sprachsteuerung ist die Kalibrierung von System zu System verschieden, allerdings nur mit einem sehr geringen Zeitaufwand verbunden. Da die Verwendung einer Muskelsteuerung oder einer Steuerung durch die Gehirnaktivität durch Trial und Error geprägt ist, erfordert auch die Kalibrierung zu Beginn und vor jeder Benutzung einen größeren Aufwand.
\newpage
%
%first one
\begin{longtable}{|p{1.8cm}|p{1.3cm}|p{1.8cm}|p{1.6cm}|p{1.5cm}|p{1.7cm}|p{2cm}|}
\caption{Objektive Kriterien der Eingabemethoden}\\
\hline
\textbf{ } & \textbf{Kosten} & \textbf{Räumlich- keiten} & \textbf{Größe} & \textbf{Gewicht} & \textbf{Wartung und Reinigung} & \textbf{Grund- voraussetzung}\\
\hline
\endfirsthead
\multicolumn{7}{c}%
{\tablename\ \thetable\ -- \textit{Objektive Kriterien der Eingabemethoden}} \\
\hline
\textbf{ } & \textbf{Kosten} & \textbf{Räumlich- keiten} & \textbf{Größe} & \textbf{Gewicht} & \textbf{Wartung und Reinigung} & \textbf{Grund- voraussetzung}\\
\hline
\endhead
\hline \multicolumn{7}{r}{\textit{Wird auf der nächsten Seite weitergeführt}} \\
\endfoot
\hline
\endlastfoot
\textbf{Sprach- steuerung}&je nach Anwendungsfall&keine Einschränkung&Mikro- phons- und Interaktionsgerätsgröße&kein zusätzliches Gewicht&Wartung, wenn sich die Stimme verändert&klare \newline Sprache\\ \hline
\textbf{Augen- steuerung}&100-13.500€&drinnen&richtet sich nach Interaktionsgerät&312g bis 8.9kg&nein&kontrollierte Augenbewegungen\\ \hline
\textbf{Gesten- steuerung}&400-2000€&keine Einschränkung&fausgroß und größer&gering (Gewicht des Joysticks)&Reinigung bei Mundsteuerung&kontrollierte Bewegungen der einzelnen Körperteile\\ \hline
\textbf{Muskel- steuerung}&250€&Myo-Armband keine Einschränkung&Myo-Armband: 11.9 x 7.4 x 10.4 cm&Myo-Armband: 254g&nein&Kontrolle über Muskelan- und entspannungen\\ \hline
\textbf{Steuerung durch Gehirn- aktivität}&720€&EPOC+ keine Einschränkung&EPOC+: 23 x 38 x 38 cm&EPOC+: 540g&nein&selbssttändi- ge Aktivierung der Signale
\label{tab:matrixObj} 
\end{longtable}
%
\newpage
%%second one
\begin{longtable}{|p{1.8cm}|p{1.65cm}|p{1.1cm}|p{1.3cm}|p{1.5cm}|p{0.95cm}|p{3.4cm}|}
\caption{Subjektive Kriterien der Eingabemethoden}\\
\hline
\textbf{ } & \textbf{Dauer} & \textbf{Kogni- tive Anstrengung} & \textbf{Körper- liche Anstrengung} & \textbf{UX} & \textbf{Kali- brierung} & \textbf{Sonstiges}\\
\hline
\endfirsthead
\multicolumn{7}{c}%
{\tablename\ \thetable\ -- \textit{Subjektive Kriterien der Eingabemethoden}} \\
\hline
\textbf{ } & \textbf{Dauer} & \textbf{Kogni- tive Anstrengung} & \textbf{Körper- liche Anstrengugn} & \textbf{UX} & \textbf{Kali- brierung} & \textbf{Sonstiges}\\
\hline
\endhead
\hline \multicolumn{7}{r}{\textit{Wird auf der nächsten Seite weitergeführt}} \\
\endfoot
\hline
\endlastfoot
\textbf{Sprach- steuerung}&keine Einschränkung&niedrig&keine&intuitiv&teils&laut und deutlich sprechen, normale Sprechgeschwindig- keit, keine zu große Distanz zum Mikrophon\\ \hline
\textbf{Augen- steuerung}&beschränkt&niedrig&hoch&gewöh- nungsbedürftig&ja&kritisch bei Brille und unterschiedlichen Profilen\\ \hline
\textbf{Gesten- steuerung}&keine Einschränkung&niedrig&niedrig&intuitiv&teils&genaue Anpassung bei ruckartigen Bewegungen\\ \hline
\textbf{Muskel- steuerung}&beschränkt&hoch&hoch&gewöh- nungsbedürftig&ja&genaue Positionierung der Elektroden\\ \hline
\textbf{Steuerung durch Gehirn- aktivität}&beschränkt&hoch&niedrig&gewöh- nungsbedürftig&ja&messbare Gehirnsignale über 3.5 Hz
\label{tab:matrixSubj} 
\end{longtable}

\section{Einsatzgebiete}
%
%
%TODO continue here
Es gibt viele verschiedene Bereiche in denen die in Kapitel ~\ref{cha:Ausgabe} beschriebene alternativen Eingabemethoden eingesetzt werden.
\newline \newline
Es gibt sehr viele verschiedene Einsatzgebiete für Sprachsteuerungssysteme. Es verschiedene Sprachassistenten wie \zB Siri oder Cortana, mit denen ein Computer zum Teil mit Hilfe der Sprache gesteuert werden kann.
Des weiteren können Navigationssysteme im Auto über Spracheingabe gesteuert werden, jedoch mit eingeschränkten Funktionen. Momentaner Trend ist allerdings Alexa bzw. der Amazon Echo Dot. Dabei handelt es sich um einen Lautsprecher inklusive Software, mit dem über Sprache kommuniziert wird. Das System kann Musik von verschiedenen Plattformen abspielen, Telefonate starten, Nachrichten schicken, Nachrichten beantworten oder vorlesen und kann Lampen bzw. Lichter im Haus steuern \cite{Alexa}. 
\newline \newline \newline \newline
Augensteuerungen können als alternative PC-Steuerung, bei Spielanwendungen oder als Steuerung für Smartphones eingesetzt werden. Das Unternehmen The Eye Tribe%
\footnote{theeyetribe.com}
%
entwickelte eine Erweiterung für Android Smartphones. Für die Benutzung muss ein USB-Ähnliches Modul an das Smartphone angesteckt werden. Hinauf- und Hinunterscrollen ist vollständig über die Augen steuerbar, für die Auswahl von Elementen bzw. Menüpunkten, muss eine zusätzlicher Touch am Bildschrim erfolgen \cite{eyeTribe}. Die Augensteuerung der Firma Humanelektronik verschafft darüber hinaus Menschen, die Probleme haben sich sprachlich auszudrücken, ein leistungsfähiges Kommunikationswerkzeug \cite{SEETECH}.
\newline \newline
Gestensteuerung findet vor allem Anwendung in der assistierenden Technologie. Abb.~\ref{fig:mund}(a) zeigt die IntegraMouse Plus, die die Produktion von Musik, das Computerspielen, die alltägliche Arbeit, aber vor allem eine selbständige und unabhängige Benutzung des Computers ermöglicht \cite{INTEGRA_Stories}. Die in Abb.~\ref{fig:kinn}(a)(b) dargestellten Kinnsteuerungen werden in der Praxis dazu verwendet, einen Rollstuhl selbstständig zu bedienen. Es können so die Fahrtrichtung gesteuert und durch das Menü navigiert werden.
\newline \newline
Das in Abb.~\ref{fig:MyoBand} dargestellte Myo-Armand, das durch Muskelan- und entspannung gesteuert wird, wird durch den Einsatz der verschiedenen Gesten als Präsentationswerkzeug verwendet, um beispielsweise Folien vor oder zurückzuschalten, Bilder zu zoomen oder den Mauszeiger zu bewegen. Darüber hinaus kann im Internet gesurft, Musik abgespielt und Computerspiele können gespielt werden \cite{myoBand}.
\newline \newline
Abb.~ref{fig:epoc} zeigt das EPOC+. Durch erhöhte Gehirnaktivität in den verschiedenen Arealen des Gehirns kann ein Spielcharakter durch eine virtuelle Welt bewegt werden. Darüber hinaus gibt es die Möglichkeit als Benutzer ferngesteuerte Autos oder Hubschrauber zu steuern \cite{epoc}.
\newline \newline \newline
Zusammenfassend lässt sich sagen, dass es viele Faktoren bei der Verwendung der verschiedenen Eingabemethoden zu beachten gilt. Je nach Anwendungsbereich, Räumlichkeiten und vor allem Benutzer sollten daher die alternativen Systeme vorher gründlich überprüft werden, bevor sie ausgewählt werden. Darüber hinaus gibt es  bereits viele verschiedene Produkte bzw. Einsatzgebiete.
