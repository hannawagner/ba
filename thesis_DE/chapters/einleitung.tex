\chapter{Einleitung}
\label{cha:Einleitung}

Die meisten Interaktionen mit elektronischen Geräten jeder Art finden durch den Einsatz der Hände statt. Das reicht von der einfachen Bedienung von Geräten (\zB Einschalten eines Computers) bis hin zur Steuerung von Abläufen mittels Gesten (\zB Liederwechsel durch eine Wischgeste, die von einer Software erkannt und in einen Befehl umgewandelt wird). Doch nicht immer gibt es die Möglichkeit mit Hilfe der Hände zu interagieren. Beispielsweise sollen beim Autofahren beide Hände am Lenkrad bleiben, bei anstrengenden und präzisen Arbeiten an Maschinen müssen oft beide Hände benutzt werden oder aber auch für Menschen mit Tetraplegie bzw. Tetraparese, die ihre Hände und Arme nicht oder oft nur sehr eingeschränkt benutzen können, sind alternative Interaktionsmöglichkeiten äußerst nützlich.
\newline \newline
Kapitel ~\ref{cha:Eingabe} stellt die einzelnen Eingabemethoden vor. Diese werden einerseits aus einer technischen Sicht (wie funktioniert das System), andererseits aus der Interaktionssicht (was muss ein Benutzer für eine erfolgreiche Interaktion machen bzw. beachten) beschrieben. Alternativ zur Interaktion mit einem System mit Hilfe der Hände, gibt es die Möglichkeit durch Sprachsteuerung, Augensteuerung, Gestensteuerung, Muskelsteuerung oder durch Steuerung basierend auf Gehirnaktivität mit einem Computer zu interagieren. Bei der Sprachsteuerung wird das Gesprochene digital erfasst, mit bereits bestehenden Mustern und Wortlauten verglichen, um so den Inhalt zu erfassen und die gewünschte Aktion durchzuführen. Bei der Augensteuerung werden sowohl mit Kameras die Augenbewegungen am Bildschirm mitverfolgt, als auch gleichzeitig von einer Software in beispielsweise Mausbewegungen umgewandelt. Die Gestensteuerung kann in mehrere einzelnen Gesten aufgeteilt werden. Zusammenfassend kann jedoch gesagt werden, dass meist ein joystickähnliches Element für die Bewegungsrichtung verwendet und zusätzliche Tasten bei der Auswahl von gewünschten Symbolen oder Elementen helfen. Die Erfassung von elektromyografischen Daten bieten die Grundlage für die Muskelsteuerung. Der Benutzer kann durch gezieltes An- und Entspannen der Muskel die gewünschte Interaktion herbeiführen. Die Steuerung durch Gehirnaktivität funktioniert über die Messung der Wellen im menschlichen Gehirns mit Hilfe eines  Elektroenzephalografen. Erhöhte Gehirnaktivität in den einzelnen Bereichen kann in Befehle bzw. in eine Interaktion umgewandelt werden.
\newline \newline \newline
Anschließend werden in Kapitel~\ref{cha:Ausgabe} Ausgabemethoden vorgestellt. Ein Computer hat die Möglichkeit durch auditive, haptische oder visuelle Ausgabe dem Benutzer Feedback zu geben.
\newline \newline
In Kapitel ~\ref{cha:Vergleich} werden abschließend verschiedene Kriterien definiert und eine Matrix erstellt. Anhand dieser Kriterien werden die Vor- und Nachteile der einzelnen alternativen Eingabemöglichkeiten erläutert. Zusätzlich werden gesonderte Rahmenbedingungen vorgestellt und Einsatzgebiete zu den einzelnen Systemen beschrieben.
\newline \newline
Ziel der Arbeit ist es einen umfassenden Überblick mit samt der Vor- und Nachteile, sowie Einsatzgebiete über alternative Eingabe- und Ausgabemethoden zu geben.