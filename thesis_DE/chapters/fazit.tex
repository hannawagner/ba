\chapter{Fazit und Ausblick}

Alternative Interaktionsmöglichkeiten ohne Verwendung der Hände gewinnen an Bedeutung. Immer häufiger werden die verschiedenen Systeme, sowohl im Alltag, als assistierende Technologie oder im Arbeitsbereich, eingesetzt.
\newline \newline
In letzter Zeit nahmen Spracheingabesysteme an Beliebtheit zu, wie an dem Beispiel von Amazon Echo zu sehen ist. Des weiteren gibt es bereits einen Sprachassistenten auf dem Markt, der von Google entwickelt wurde. Der Google Assistent%
\footnote{https://assistant.google.com/}
%
hat die selben Ziele, wie das System von Amazon, nämlich Fragen zu beantworten, Aufgaben zu übernehmen und technische Geräte im Haus zu steuern. Laut Futurezone \cite{Futurezone} wird auch das Unternehmen Apple im nächstes Jahr einen Lautsprecher mit digitaler Assistenz vorstellen. Das Grundkonzept baut auf dem bereits existierenden Sprachassistenten Siri auf.
\newline \newline
Obwohl es schon viele verschiedene alternative Systeme für die Interaktion mit einem Computer gibt, besteht dennoch Verbesserungspotential. Bei den meisten Eingabesystemen ist die Fragen von zentraler Bedeutung, wie bestimmte Fehler korrigiert (\zB Lichtprobleme bei der Augensteuerung) bzw. die Leistungsfähigkeit der Systeme gesteigert werden können. Hingegen liegt bei der Steuerung durch Muskel- und Gehirnaktivität der Fokus auf der Herstellung von mehr marktreifen Geräten und wie die Signale bei nicht invasiven Systemen verstärkt werden können. Das Unternehmen Facebook%
\footnote{www.facebook.com}
%
arbeitet beispielsweise an einem System, das Benutzern ermöglichen soll, 100 Wörter pro Minute durch reine Gedankensteuerung eingeben zu können \cite{Facebook}.
\newline \newline
Die Frage, die man sich in Zukunft, stellen muss, ist, wie die rechtlichen Rahmenbedienungen ausschauen. Gibt es ein beispielsweise ein ausreichend großes Zielpublikum, das sich von einer Software die Gedanken lesen lässt. Wer entscheidet, welche Informationen für das Generieren eines Textes relevant sind und wer besitzt Zugang zu diesen Daten? Welche Daten werden bei einem Sprachassistenten, der permanent zuhört, gespeichert bzw. wie werden sie verwendet? Um eine größere Akzeptanz bzw. ein größeres Zielpublikum erreichen zu können, müssen diese Fragen ausreichend und transparent geklärt werden.
