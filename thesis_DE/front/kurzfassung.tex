\chapter{Kurzfassung}
Diese Arbeit gibt einen Einblick in verschiedene alternative Ein- und Ausgabemethoden, vergleicht die einzelnen Methoden miteinander und beschreibt deren Einsatzgebiete.
\newline \newline
Es gibt viele Anwendungsfälle, in denen die Interaktionen mit einem System mit den
Händen nicht möglich ist. In alltäglichen Situation, wie beispielsweise beim Autofahren,
bei anstrengenden und präzisen Arbeiten an Maschinen oder für Menschen mit
Beeinträchtigung, sind Alternativen notwendig.
\newline \newline
Alternativ zur Interaktion mit einem System mit Hilfe der Hände, gibt es die Möglichkeit durch Sprachsteuerung, Augensteuerung, Gestensteuerung, Muskelsteuerung oder durch Steuerung basierend auf Gehirnaktivität mit einem Computer zu interagieren. Einige Systeme müssen für eine erfolgreiche Interaktion ausreichend trainiert und kalibriert werden, andere könne sofort verwendet werden. Aber nicht nur die Vorbereitung dauert unterschiedlich lange, sondern auch die Interaktion an sich ist verschieden. Einige Systeme verwenden ein joystickähnliches Element als Basis, andere eine Kamera und wieder andere verwenden Sensoren.
\newline \newline
Ein Computer hat im Gegenzug die Möglichkeit durch auditive, haptische oder visuelle Ausgabe dem Benutzer Feedback geben zu können. Je nach Anwendungsfall werden die verschiedenen Möglichkeiten eingesetzt.
\newline \newline
Alternative Interaktionsmöglichkeiten finden Einsatz in alltäglichen Situationen, im Arbeitsleben und als assistive Technologien. Auf Grund des momentanen Beliebtheitsgrades von Amazon Alexa und weiteren Sprachassistenten wurde die Sprachausgabe als die beliebte Art alternativen Interaktionssystem im Moment definiert. 