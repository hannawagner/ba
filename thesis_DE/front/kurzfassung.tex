\chapter{Kurzfassung}

%kurze einleitung
Die folgende Arbeit gibt einen Einblick in verschiedene alternative Ein- und Ausgabemethoden, vergleicht die einzelnen Systeme miteinander und beschreibt deren Einsatzgebiete.
\newline \newline
%die fünf eingabemethoden, ev kurz beschreiben wie sie funktioniren
Alternativ zur Interaktion mit einem Systems mit Hilfe der Hände, gibt es die Möglichkeit durch Sprachsteuerung, Augensteuerung, Gestensteuerung, Muskelsteuerung oder durch Steuerung basierend auf Gehirnaktivität mit einem Computer zu interagieren. Bei der Sprachsteuerung wird das Gesprochene digital erfasst, mit bereits bestehenden Mustern und Wortlauten verglichen, um so den Inhalt zu erfassen und die gewünschte Aktion durchzuführen. Bei der Augensteuerung wird sowohl mit Kameras die Augenbewegungen am Bildschirm mitverfolgt, als auch gleichzeitig von einer Software in beispielsweise Mausbewegungen parallel umgewandelt. Die Gestensteuerung kann in mehrere einzelnen Gesten aufgeteilt werden. Zusammenfassend kann jedoch gesagt werden, dass meist ein joystickähnliches Element für die Bewegungsrichtung verwendet und zusätzliche Tasten bei der Auswahl von gewünschten Symbolen oder Elementen helfen. Die Erfassung von elektromyografischen Daten (EMG) bieten die Grundlage für die Muskelsteuerung. Der Benutzer kann durch gezieltes An- und Entspannen der Muskel die gewünschte Interaktion herbeiführen. Die Steuerung durch Gehirnaktivität funktioniert über die Messung der Wellen im menschlichen Gehirns mit Hilfe eines  Elektroenzephalografen (EEG). Erhöhte Gehirnaktivität in den einzelnen Bereichen kann in Befehle bzw. in eine Interaktion umgewandelt werden.
\newline \newline
%ausgabe
Darüber hinaus werden verschiedene Ausgabemethoden behandelt. Ein Computer hat die Möglichkeit durch auditive, haptische oder visuelle Ausgabe dem Benutzer Feedback geben zu können.
\newline \newline
%vergeleich, einsatzgebiete
Alternative Interaktionsmöglichkeiten finden Einsatz in alltäglichen Situationen, im Arbeitsleben und als assistive Technologien. Auf Grund des momentanen Beliebtheitsgrades von Amazon Alexa und weiteren Sprachassistenten wurde die Sprachausgabe als die beliebte Art alternativen Interaktionssystem im Moment definiert. Hingegen werden bei der Steuerung durch Gehirnaktivität die meisten Potentiale gesehen.
