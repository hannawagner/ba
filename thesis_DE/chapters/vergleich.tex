\chapter{Gegenüberstellung der Methoden}
\label{cha:Vergleich}

In Kapitel ~\ref{cha:Ausgabe} wurden die verschiedenen Eingabemethoden näher erläutert. Dieses Kapitel beschäftigt sich damit, welche Vor- und Nachteile, gesonderte Rahmenbedingungen und welche Einsatzgebiete es zu den einzelnen Methoden gibt. Da der Fokus dieser Arbeit auf der Steuerung von Systemen liegt, wird in diesem Kapitel auf die verschiedenen Ausgabemethoden nicht eingegangen.

\section{Fördernde und hemmende Faktoren}
%
Jede einzelne Eingabemethode weist gewisse Pro- und Contraseiten auf. Es wurden verschiedene Kriterien für einen Vergleich definiert. Tab.~\ref{tab:matrixObj} zeigt die objektiven Kriterien, Tab.~\ref{tab:matrixSubj} stellt die subjektiven Kriterien dar.
\newline \newline
Der erste Vergleichsfaktor, der in Tab.~\ref{tab:matrixObj} dargestellt ist, bezieht sich auf die Kosten. Bei der Sprachsteuerung gibt es verschiedene Sprachassistenten wie z.B. Siri%
\footnote{https://www.apple.com/ios/siri/}
%
 oder Cortana%
\footnote{https://support.microsoft.com/en-us/help/17214}
%
 , mit denen ein Computer zum Teil mit Hilfe der Sprache gesteuert werden kann. Die Programme ansich sind kostenlos, wenn der Benutzer über das jeweilige Gerät verfügt (IPhone etc. oder Computer mit Windows10).
Wie in Abschnitt \label{cha:Augensteuerung} erklärt, werden für die Augensteuerung sowohl ein Eye-Tracking-System, als auch eine Software benötigt. Die in Abb.~\ref{fig:Tobii}(a) belaufen sich beispielsweise auf rund 13.500€, wobei hier noch die Kosten für die Software hinzukommen \cite{TobiiCosts}. Allerdings kann auch eine günstigere Variante implementiert werden. Hierbei benötigt man zwei Webcams (ca.100€) und eine Open Source Software für die Blickerkennung und Steuerung. Kinn- und Mundsteuerungen kosten zwischen 400€ und 2000€ \cite{SENSORY} \cite{INTEGRA}. Das Myo-Armband als Referenz für die Muskelsteuerung beläuft sich auf ca. 250€, wobei komplexere Systeme wesentlich teurer sind \cite{myoBand}. Das Epoc+ des Unternehmens Emotiv%
\footnote{https://www.emotiv.com}
%
, das dem Benutzer erlaubt mit einem System mit Hilfe von Gehirnaktivität zu interagieren, kostet rund 720€.
\newline \newline \newline \newline
Bei den meisten System gibt es keine räumliche Einschränkung und sie sind daher nicht an ein gewisses Setting gebunden. Bei der Benutzung einer Augensteuerung ist allerdings darauf zu achten, dass keine störenden Lichtquellen die Kamera bzw. die Interaktion behindern. Des weiteren ist ein Benutzer bei der Verwendung von Systemen, die durch die Gehirn- oder Muskelaktivität gesteuert werden, oft an notwendige Geräte und somit an gewisse Räumlichkeiten gebunden.
\newline \newline
Größe
\newline \newline
Gewicht
\newline \newline
Wartung
\newline \newline
Für die Benutzung jeder der verschiedenen Eingabemethoden gelten die selben Grundvoraussetzungen. Der Benutzer muss eine eigenständige Kontrolle über das jeweilige Eingabemittel (Sprache, Muskelanspannung, Augenbewegungen etc.) besitzen, damit eine Interaktion möglich ist. Zusätzlich gibt es spezifische Faktoren wie z.B. ist bei der Sprachsteuerung eine laute und deutliche Sprache ein fördernder Faktor, hingegen erweisen sich verstaubte oder spiegelnde Brillengläser bei der Augensteuerung als ein hemmender Faktor.
\newline \newline
Auch die Interaktionsdauer mit dem System ist sehr unterschiedlich. Da die Sprachsteuerung und die Gestensteuerung nur eine geringe kognitive und körperliche Anstrengung erfordern, gibt es bei der Benutzungsdauer kaum Einschränkungen. Im Gegensatz dazu findet man bei der Augensteuerung eine hohe körperliche Anstrengung und sollte daher nicht zu lange benutzt werden, damit die Augen langfristig nicht geschädigt werden. Ähnlich als bei der gewöhnlichen Benutzung eines Computers sollte alle 20 Minuten eine Pause eingelegt werden \cite{20Methode}. Das Benutzen einer Steuerung, die durch die Muskeln oder durch die Gehirnaktivität funktioniert, ist mit hoher kognitiven Anstrengung verbunden. Zusätzlich erfordert das Anspannen und Entspannen der Muskelpartien eine hohe körperliche Anstrengung. Aus diesen Gründen ist die Interaktionsdauer der beiden Methoden sehr beschränkt. Daher muss während der Interaktion regelmäßig überprüft werden, ob zum gegebenen Zeitpunkt eine Überanstrengung erfolgt. Ist dies der Fall sollte eine Pause eingelegt werden, um den einzelnen Partien eine Erholungsphase zu ermöglichen.
\newline \newline
Kognitive Anstrengung
\newline \newline
Körperliche Anstrengung
\newline \newline
UX
\newline \newline
Bevor ein Benutzer mit einem System interagieren kann, muss ein Gerät oft individuell angepasst (kalibriert) werden. Dies ist bei allen beschriebenen Eingabemethoden, ausschließlich der Gestensteuerung, der Fall. Bei der Sprachsteuerung ist die Kalibrierung von System zu System verschieden, allerdings ist der Zeitaufwand hier sehr gering. Da die Verwendung einer Muskelsteuerung oder einer Steuerung durch die Gehirnaktivität durch Trial und Error geprägt ist, ist auch die Kalibrierung zu Beginn und vor jeder Benutzung sehr aufwendig.
\newline \newline
Sonstiges
%
%
%
%TODO make two tables here
%first one
\begin{longtable}{|p{1.8cm}|p{1.4cm}|p{1.8cm}|p{1.5cm}|p{1.5cm}|p{1.7cm}|p{2cm}|}
\caption{Objektive Faktoren der Eingabemethoden}\\
\hline
\textbf{ } & \textbf{Kosten} & \textbf{Räumlich- keiten} & \textbf{Größe} & \textbf{Gewicht} & \textbf{Wartung und Reinigung} & \textbf{Grund- voraussetzung}\\
\hline
\endfirsthead
\multicolumn{7}{c}%
{\tablename\ \thetable\ -- \textit{Objektive Kategorien}} \\
\hline
\textbf{Objektive Kategorien} & \textbf{Kosten} & \textbf{Räumlichkeiten} & \textbf{Größe} & \textbf{Gewicht} & \textbf{Wartung und Reinigung} & \textbf{Grundvoraussetzung}\\
\hline
\endhead
\hline \multicolumn{7}{r}{\textit{Wird auf der nächsten Seite weitergeführt}} \\
\endfoot
\hline
\endlastfoot
\textbf{Sprach- steuerung}&je nach Anwendungsfall&keine Einschränkung&Mikro- phons- und Interaktionsgerätsgröße&kein zusätzliches Gewicht&Wartung, wenn sich die Stimme verändert&klare Sprache\\ \hline
\textbf{Augen- steuerung}&100-13.500€&drinnen&richtet sich nach Interaktionsgerät&312g bis 3.8kg&nein&kontrollierte Augenbewegungen\\ \hline
\textbf{Gesten- steuerung}&400-2000€&keine Einschränkung&Fausgroß und größer&gering (Gewicht des Joysticks)&Reinigung bei Mundsteuerung&kontrollierte Bewegungen der einzelnen Körperteile\\ \hline
\textbf{Muskel- steuerung}&250€&Myo-Armband keine Einschränkung&Myo-Armband: 11,9 x 7,4 x 10,4 cm&Myo-Armband: 254g&nein&Kontrolle über Muskelan- und entspannungen\\ \hline
\textbf{Steuerung durch Gehirn- aktivität}&720€&Labor- setting&sehr klein&sehr gering&nein&Selbssttändi- ge Aktivierung der Signale
\label{tab:matrixObj} 
\end{longtable}
%
%%second one
\begin{longtable}{|p{1.8cm}|p{1.65cm}|p{1.1cm}|p{1.3cm}|p{1.5cm}|p{0.95cm}|p{3.4cm}|}
\caption{Subjektive Faktoren der Eingabemethoden}\\
\hline
\textbf{ } & \textbf{Dauer} & \textbf{Kogni- tive Anstrengung} & \textbf{Körper- liche Anstrengung} & \textbf{UX} & \textbf{Kali- brierung} & \textbf{Sonstiges}\\
\hline
\endfirsthead
\multicolumn{7}{c}%
{\tablename\ \thetable\ -- \textit{Subjektive Kategorien}} \\
\hline
\textbf{ } & \textbf{Dauer} & \textbf{Kognitive Anstrengung} & \textbf{Körperliche Anstrengugn} & \textbf{UX} & \textbf{Kalibrierung} & \textbf{Sonstiges}\\
\hline
\endhead
\hline \multicolumn{7}{r}{\textit{Wird auf der nächsten Seite weitergeführt}} \\
\endfoot
\hline
\endlastfoot
\textbf{Sprach- steuerung}&keine Einschränkung&niedrig&keine&intuitiv&teils&laut und deutlich sprechen, normale Sprechgeschwindigkeit, keine zu Große Distanz zum Mikrophon\\ \hline
\textbf{Augen- steuerung}&beschränkt&niedrig&hoch&gewöh- nungsbedürftig&ja&Kritisch bei Brille und unterschiedlichen Profilen\\ \hline
\textbf{Gesten- steuerung}&keine Einschränkung&niedrig&niedrig&intuitiv&teils&Genaue Anpassung bei ruckartigen Bewegungen\\ \hline
\textbf{Muskel- steuerung}&beschränkt&hoch&hoch&gewöh- nungsbedürftig&ja&Genaue Positionierung der Elektroden\\ \hline
\textbf{Steuerung durch Gehirn- aktivität}&beschränkt&hoch&niedrig&gewöh- nungsbedürftig&ja&messbare Gehirnsignale über 3.5 Hz
\label{tab:matrixSubj} 
\end{longtable}
\newpage
\section{Einsatzgebiete}
%
%
%TODO continue here
Es gibt viele verschiedene Bereiche in denen die in Kapitel ~\ref{cha:Ausgabe} beschriebene alternativen Eingabemethoden eingesetzt werden.
\newline \newline
Zum einen werden viele Steuerungen als assistive Technologie für Menschen mit Beeinträchtigung eingesetzt. Besonders für Menschen mit Tetraplegie bzw. Tetraparese, die ihre Hände und Arme nicht oder oft nur sehr eingeschränkt benutzen können, sind die Einsatzgebiete relevant. Abb.~\ref{fig:mund}(a) zeigt die IntegraMouse Plus, die die Produktion von Musik, das Computerspielen, die alltägliche Arbeit, aber vor allem eine selbständige und unabhängige Benutzung des Computers ermöglicht \cite{INTEGRA_Stories}. Die Augenstreuung der Firma Humanelektronik verschafft Menschen, die Probleme haben sich sprachlich auszudrücken, ein leistungsfähiges Kommunikationswerkzeug \cite{SEETECH}. Die in Abb.~\ref{fig:kinn}(a)(b) dargestellten Kinnsteuerungen werden in der Praxis dazu verwendet, einen Rollstuhl selbstständig zu bedienen. Es können so die Fahrtrichtung gesteuert und durch das Menü navigiert werden.
\newline \newline
Es gibt aber auch viele alltägliche Situationen, in denen Systeme eingesetzt werden, die ohne die Hände zu bedienen sind. Beispielsweise funktionieren Freisprechanlage oder Navigationssysteme im Auto bereits über Spracheingabe. Momentaner Trend ist allerdings Alexa bzw. der Amazon Echo Dot. Dabei handelt es sich um einen Lautsprecher inklusive Software, mit dem über Sprache kommuniziert wird. Das System kann Musik von verschiedenen Plattformen abspielen, Telefonate starten, Nachrichten schicken, Nachrichten beantworten oder vorlesen und kann Lampen bzw. Lichter im Haus steuern \cite{Alexa}. 
\newline \newline
Aber nicht nur im privaten, sondern auch im beruflichen Bereich werden oft alternative Eingabemethoden eingesetzt. Das in Abb.~\ref{fig:MyoBand} dargestellte Myo-Armand wird durch den Einsatz der verschiedenen Gesten als Präsentationswerkzeug verwendet, um beispielsweise Folien vor oder zurückzuschalten \cite{myoBand}.
\newline \newline \newline
Zusammenfassend lässt sich sagen, dass es viele Faktoren bei der Verwendung der verschiedenen Eingabemethoden zu beachten gilt. Je nach Anwendungsbereich, Räumlichkeiten und vor allem Benutzer sollten daher die alternativen Systeme vorher gründlich überprüft werden, bevor sie ausgewählt werden. Darüber hinaus gibt es ausschließlich der Steuerung durch Gehirnaktivität bereits viele verschiedene Produkte bzw. Einsatzgebiete.
