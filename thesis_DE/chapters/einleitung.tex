\chapter{Einleitung}
\label{cha:Einleitung}

Die meisten Interaktionen mit elektronischen Geräten jeder Art finden durch den \mbox{Einsatz} der Hände statt. Das reicht von der einfachen Bedienung von Geräten (\zB Einschalten eines Computers) bis hin zur Steuerung von Abläufen mittels Gesten (\zB Liederwechsel durch eine Wischgeste, die von einer Software erkannt und in einen Befehl umgewandelt wird). Doch nicht immer kann mit Hilfe der Hände interagiert werden. 
In vielen Alltagssituationen bedarf es oft beider Hände, beispielsweise beim Autofahren oder bei anstrengenden und präzisen Arbeiten an Maschinen. Aber auch für Menschen mit \mbox{Tetraplegie} bzw. Tetraparese, die ihre Hände und Arme nicht oder oft nur sehr eingeschränkt benutzen können, sind alternative Interaktionsmöglichkeiten von Vorteil.
\newline \newline
Kapitel ~\ref{cha:Eingabe} stellt die einzelnen Eingabemethoden vor. Diese werden einerseits aus technischer Sicht (wie funktioniert das System), andererseits aus Interaktionssicht (was muss ein Benutzer%
\footnote{Aus Gründen der besseren Lesbarkeit wird in dieser Arbeit die Sprachform des generischen Maskulinums angewendet. Es wird an dieser Stelle darauf hingewiesen, dass die ausschließliche Verwendung der männlichen Form geschlechtsunabhängig verstanden werden soll.}
%
 für eine erfolgreiche Interaktion machen bzw. beachten) beschrieben. \mbox{Alternativ} zur Interaktion mit einem System mit Hilfe der Hände gibt es die Möglichkeit durch Sprachsteuerung, Augensteuerung, Gestensteuerung, Muskelsteuerung oder durch Steuerung basierend auf Gehirnaktivität mit einem Computer zu interagieren. Bei der Sprachsteuerung wird das Gesprochene digital erfasst, mit bereits bestehenden Mustern und Wortlauten verglichen, um so den Inhalt zu erfassen und die gewünschte Aktion durchzuführen. Bei der Augensteuerung werden mit Kameras die Augenbewegungen am Bildschirm mitverfolgt und beispielsweise gleichzeitig von einer Software in Mausbewegungen umgewandelt. Die Gestensteuerung kann in mehrere Arten unterteilt werden. Für alle gilt, dass meist ein joystickähnliches Element für die Bewegungsrichtung verwendet wird und zusätzliche Tasten bei der Auswahl von gewünschten Symbolen oder Elementen helfen. Die Erfassung von elektromyografischen Daten bietet die Grundlage für die Muskelsteuerung. Der Benutzer kann durch gezieltes An- und Entspannen der Muskeln die gewünschte Interaktion herbeiführen. Die Steuerung durch Gehirnaktivität funktioniert über die Messung der Wellen im menschlichen Gehirn mit Hilfe eines  Elektroenzephalografen. Erhöhte Gehirnaktivität in einzelnen Bereichen kann in \mbox{Befehle} bzw. in eine Interaktion umgewandelt werden.
\newline \newline
Anschließend werden in Kapitel~\ref{cha:Ausgabe} Ausgabemethoden vorgestellt. Ein Computer hat die Möglichkeit durch auditive, haptische oder visuelle Ausgabe dem Benutzer Feedback zu geben.
\newline \newline
In Kapitel ~\ref{cha:Vergleich} werden abschließend verschiedene Kriterien definiert und eine Matrix erstellt. Anhand dieser Kriterien werden die Vor- und Nachteile der einzelnen alternativen Eingabemöglichkeiten erläutert. Zusätzlich werden gesonderte Rahmenbedingungen vorgestellt und Einsatzgebiete zu den einzelnen Systemen beschrieben.
\newline \newline
Ziel der Arbeit ist es, einen umfassenden Einblick (einschließlich Vor- und Nachteile sowie Einsatzgebiete) über alternative Eingabe- und Ausgabemethoden zu geben.
