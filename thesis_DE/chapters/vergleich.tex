\chapter{Gegenüberstellung der Methoden}
\label{cha:Vergleich}

In Kapitel ~\ref{cha:Ausgabe} wurden die verschiedenen Eingabemethoden näher erläutert. Dieses Kapitel beschäftigt sich damit, welche Vor- und Nachteile, gesonderte Rahmenbedingungen und welche Einsatzgebiete es zu den einzelnen Methoden gibt.

\section{Fördernde und hemmende Faktoren}
%
Jede einzelne Eingabemethode weist gewisse Pro- und Contraseiten auf. Daher wurde, wie in Tab.~\ref{tab:matrix} dargesellt, eine Übersicht bzw. ein Vergleich erstellt.
\newline \newline
Ein wichtiger Punkt, den es vorab zu klären gibt, ob ein System verwendet werden soll, bezieht sich auf die Kosten. Bei der Sprachsteuerung gibt es hierfür keine Pauschalaussage, sondern es ist je nach Anwendungsfall unterschiedlich. Es gibt einige Sprachassistenten wie z.B. Siri%
\footnote{https://www.apple.com/ios/siri/}
%
 oder Cortana%
\footnote{https://support.microsoft.com/en-us/help/17214}
%
 , mit denen ein Computer zum Teil mit Hilfe der Sprache gesteuert werden kann. Wie in Abschnitt \label{cha:Augensteuerung} erklärt, werden für die Augensteuerung sowohl ein Eye-Tracking-System, als auch eine Software benötigt. Die in Abb.~\ref{fig:Tobii}(a) belaufen sich beispielsweise auf rund 13.500€, wobei hier noch die Kosten für die Software hinzukommen \cite{TobiiCosts}. Kinn- und Mundsteuerungen sind um einiges billiger und kosten zwischen 400€ und 2000€ \cite{SENSORY} \cite{INTEGRA}. Das Myo-Armband als Referenz für die Muskelsteuerung beläuft sich auf ca. 250€, wobei komplexere Systeme wesentlich teurer sind \cite{myoBand}. 
\newline \newline
Auch die Interaktionsdauer mit dem System ist sehr unterschiedlich. Da es die Sprachsteuerung und die Gestensteuerung nur eine minimale bis geringe kognitive und körperliche Anstrengung erfordert, gibt es bei der Benutzungsdauer kaum Einschränkungen. Im Gegensatz findet man bei der Augensteuerung sowohl eine körperliche, als auch eine kognitiven Anstrengung und sollte daher nicht zu lange benutzt werden, damit die Augen langfristig nicht geschädigt werden. Ähnlich wie bei der gewöhnlichen Benutzung eines Computers sollte alle 20 Minuten eine Pause eingelegt werden \cite{20Methode}. Das Benutzen einer Steuerung, die durch die Muskeln oder durch die Gehirnaktivität funktioniert, ist mit hoher kognitiven Anstrengungen verbunden. Zusätzlich erfordert das Anspannen und Entspannen der Muskelpartien eine sehr hohe körperliche Anstrengung. Aus diesen Gründe ist die Interaktionsdauer der beiden Methoden sehr beschränkt. Daher muss während der Interaktion überprüft werden, ab welchem Zeitpunkt eine Überanstrengung erfolgt und danach Pausen einteilen.
\newline \newline
Ausschließlich der Gestensteuerung müssen alle Eingabegeräte vor der ersten Benutzung individuell angepasst (kalibriert) werden. Bei der Sprachsteuerung ist dies von System zu System verschieden, allerdings ist der Zeitaufwand hier sehr gering. Da das Verwenden einer Muskelsteuerung oder einer Steuerung durch die Gehirnaktivität durch Trial und Error geprägt ist, ist auch die Kalibrierung zu Beginn und vor jeder Benutzung sehr aufwendig.
\newline \newline
Bei den meisten System gibt es keine räumliche Einschränkung und sind daher nicht an ein gewisses Setting gebunden. Bei der Benutzung einer Augensteuerung gibt es allerdings zu beachten, dass keine störenden Lichtquellen die Kamera bzw. die Interaktion behindern. Des weiteren ist man bei der Verwendung von Systemen, die durch die Gehirn- oder Muskelaktivität gesteuert werden, oft an notwendige Geräte und somit an gewisse Räumlichkeiten Gebunden.
\newline \newline
Für die Benutzung jeder der verschiedenen Eingabemethoden gelten die selben Grundvoraussetzungen. Der Benutzer muss eine eigenständige Kontrolle über das jeweilige Eingabemittel (Sprache, Muskelanspannung, Augenbewegungen etc.) besitzen, damit eine Interaktion möglich ist. Zusätzlich gibt es spezifische Faktoren wie z.B. ist bei der Sprachsteuerung eine laute und deutliche Sprache ein fördernder Faktor, hingegen erweisen sich verstaubte spiegelnde Brillengläser bei der Augensteuerung als ein hemmender Faktor.
%
%
%
%
%
%
\begin{landscape}
\begin{longtable}{|p{3cm}|p{3cm}|p{3cm}|p{3cm}|p{3cm}|p{3cm}|}

\caption{Verschiedene Faktoren der Eingabemethoden}\\
\hline
\textbf{             } & \textbf{Sprachsteuerung} & \textbf{Augensteuerung} & \textbf{Gestensteuerung} & \textbf{Muskelsteuerung} & \textbf{Steuerung durch Gehirnaktivität} \\
\hline
\endfirsthead
\multicolumn{6}{c}%
{\tablename\ \thetable\ -- \textit{Verschiedene Faktoren der Eingabemethoden}} \\
\hline
\textbf{   } & \textbf{Sprachsteuerung} & \textbf{Augensteuerung} & \textbf{Gestensteuerung} & \textbf{Muskelsteuerung} & \textbf{Steuerung durch Gehirnaktivität} \\
\hline
\endhead
\hline \multicolumn{6}{r}{\textit{Wird auf der nächsten Seite weitergeführt}} \\
\endfoot
\hline
\endlastfoot
Kosten                  & je nach Anwendungsfall, für Desktopsteuerung einige kostenlose Anwednungen                      & Kosten für Eye-Tracking-System und Software,  Tobii Pro Glasses 2  ca. 13.500€                                                                            & Kosten für gewöhnlichen Joystick gering, Sensory Kinnsteuerung: ca. 400€, QuadStick Gamecontroller: ca. 500€, IntegraMouse Plus: ca. 2000€ & Myo-Armband: ca. 250€                                                                    & keine Produkte am Markt                                                  \\
Dauer                   & keine Einschränkung                                                                             & beschränkt, 20-20-20-Methode                                                                                                                              & keine Einschränkung, bei optischer Methode der Kopfsteuerung ähnl. Augensteuerung                                                          & beschränkt                                                                               & beschränkt                                                               \\
Kognitive Anstrenung    & gering                                                                                          & mittel                                                                                                                                                    & gering                                                                                                                                     & hoch                                                                                     & sehr hoch                                                                \\
Körperliche Anstrengung & keine                                                                                           & hoch                                                                                                                                                      & gering                                                                                                                                     & sehr hoch                                                                                & gering                                                                   \\
UX                      & intuitiv, Sprache ist häufigstes mittel zur Kommunikation/Interaktion                           & gewöhungsbedürftig, aber dennoch intuitiv (Auswahl funktioniert über Fixation)                                                                            & intuitive bewegung mit dem Kinn, Füßen und Kopf, kurze Eingewöhnungsphase bei Mundsteuerung (nippen, pusten)                               & gewöhungsbedürftig (Trial and Error)                                                     & gewöhungsbedürftig (Trial and Error)                                     \\
Kalibrierung            & teils                                                                                           & ja                                                                                                                                                        & nein, ja bei optischer Variante der Kopfsteuerung                                                                                          & ja, aufwendig                                                                            & ja, sehr aufwendig                                                       \\
Wartung/Reinigung       & Wartung, wenn sich die Stimme verändert (z.B. Stimmbruch)                                       & nein                                                                                                                                                      & Wartung nein, Reinigung ja z.B. bei Mundsteuerung                                                                                          & nein                                                                                     & nein                                                                     \\
Räumlichkeiten          & keine Einschränkung                                                                             & drinnen (Lichteinflüsse)                                                                                                                                  & keine Einschränkung, ja bei optischer Kopfstuerung                                                                                         & Laborsetting, keine Einschränkung bei Benutzung des Myo Armbandes                        & Laborsetting                                                             \\
Größe                   & meist kein zusätzliches Gerät, ansonsten richtet sich die nach Mikrophon und Interaktionssystem & SMI Red250mobil: 24 x 2.7 x 3 cm; Tobii Pro Spectrum 55 cm x 18 cm x 6 cm; Tobii Pro Glasses 2: 130 x 85 x 27 mm;                                         & Mundsteuerungen bzw. gewöhnliche Joysticks ungefähr Faustgroß, andere Systeme etwas größer                                                 & Myo-Armband: 11,9 x 7,4 x 10,4 cm                                                        & sehr klein (Größer der Elektroden)                                       \\
Gewicht                 & kein zusatzliches Gewicht, Mikrophon meist im System integriert                                 & Tobii Pro Glasses 2:  312 g, Tobii Pro Spectrum: 3.8 kg                                                                                                   & gering (Gewicht des Joysticks)                                                                                                             & Myo-Armband: 254g                                                                        & sehr gering (Gewicht der Elektroden)                                     \\
Grundvorraussetzung     & klare Sprache                                                                                   & kontrollierte Augenbewegungen                                                                                                                             & kontrollierte Bewegungen der ausgewählten Körperteile                                                                                      & Kontrolle über Muskelan- und entspannungen                                               & Selbssttändige Aktivierung der Gehirnsignale in den zu messenden Arealen \\
Spezifische Faktoren    & laut und deutlich sprechen, normale Sprechgeschwindigkeit, keine zu Große Distanz zum Mikrophon & verstaubte spiegelnde Brillengläser sind ein hemmender Faktor, zwei unterschiedliche Profile wenn ein Benutzer zwischen Kontaktlinsen und Brille wechselt & System muss auf Benutzer genau angepasst werden, wenn die Bewegungen sehr rucksartig sind                                                  & Richtige Positionierung der Elektroden, damit genug Muskelaktivität gemessen werden kann & messbare Gehirnsignale über 3.5 Hz                                      
\label{tab:matrix} 
\end{longtable}
\end{landscape}
%
%
%
\section{Einsatzgebiete}
%
%
%TODO continue here
Es gibt viele verschiedene Bereiche in denen die in Kapitel ~\ref{cha:Ausgabe} beschriebene alternativen Eingabemethoden eingesetzt werden.
\newline \newline
Zum einen werden viele Steuerungen als assistive Technologie für Menschen mit Beeinträchtigung eingesetzt. Besonders für Menschen mit Tetraplegie bzw. Tetraparese, die ihre Hände und Arme nicht oder oft nur sehr eingeschränkt benutzen können, sind die Einsatzgebiete relevant. Abb.~\ref{fig:mund}(a) zeigt die IntegraMouse Plus, die die Produktion von Musik, das Computerspielen, die alltägliche Arbeit, aber vor allem eine selbständige und unabhängige Benutzung des Computers ermöglicht \cite{INTEGRA_Stories}. Die Augenstreuung der Firma Humanelektronik verschafft Menschen, die Probleme haben sich sprachlich auszudrücken, ein leistungsfähiges Kommunikationswerkzeug \cite{SEETECH}. Die in Abb.~\ref{fig:kinn}(a)(b) dargestellten Kinnsteuerungen werden in der Praxis dazu verwendet, einen Rollstuhl selbstständig zu bedienen. Es können so die Fahrtrichtung gesteuert und durch das Menü navigiert werden.
\newline \newline
Es gibt aber auch viele alltägliche Situationen, in denen Systeme eingesetzt werden, die ohne die Hände zu bedienen sind. Beispielsweise funktionieren Freisprechanlage oder Navigationssysteme im Auto bereits über Spracheingabe. Momentaner Trend ist allerdings Alexa bzw. der Amazon Echo Dot. Dabei handelt es sich um einen Lautsprecher inklusive Software, mit dem über Sprache kommuniziert wird. Das System kann Musik von verschiedenen Plattformen abspielen, Telefonate starten, Nachrichten schicken, Nachrichten beantworten oder vorlesen und kann Lampen bzw. Lichter im Haus steuern \cite{Alexa}. 
\newline \newline
Aber nicht nur im privaten, sondern auch im beruflichen Bereich werden oft alternative Eingabemethoden eingesetzt. Das in Abb.~\ref{fig:MyoBand} dargestellte Myo-Armand wird durch den Einsatz der verschiedenen Gesten als Präsentationswerkzeug verwendet, um beispielsweise Folien vor oder zurückzuschalten \cite{myoBand}.
\newline \newline \newline
Zusammenfassend lässt sich sagen, dass es viele Faktoren bei der Verwendung der verschiedenen Eingabemethoden zu beachten gilt. Je nach Anwendungsbereich, Räumlichkeiten und vor allem Benutzer sollten daher die alternativen Systeme vorher gründlich überprüft werden, bevor sie ausgewählt werden. Darüber hinaus gibt es ausschließlich der Steuerung durch Gehirnaktivität bereits viele verschiedene Produkte bzw. Einsatzgebiete.
