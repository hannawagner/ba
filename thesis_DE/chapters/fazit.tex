\chapter{Fazit und Ausblick}

Alternative Eingabemethoden gewinnen immer mehr an Bedeutung. Weil die verschiedenen Systeme sowohl im Alltag, als assistive Technologie oder im Arbeitsbereich eingesetzt werden, wird der Bedarf erkannt und dahingehend entwickelt.
\newline \newline
In den letzten Monaten gewannen Spracheingabesysteme immer mehr an Beliebtheit, wie an dem Beispiel von Amazon Echo zu sehen ist. Weiters gibt es bereits auch einen Sprachassistenten, der von Google entwickelt wurde. Der Google Assistent%
\footnote{https://assistant.google.com/}
%
hat die selben Ziele, wie das System von Amazon - Fragen zu beantworten, Aufgaben zu übernehmen und technische Geräte im Haus zu steuern. Laut Futurezone \cite{Futurezone} soll auch das Unternehmen Apple%
\footnote{https://www.apple.com/}
%
im nächstes Jahr einen Lautsprecher mit digitaler Assistenz herausbringen. Das Grundkonzept baut auf dem bereits existierenden Sprachassistenten Siri auf.
\newline \newline
Absehen von der Möglichkeit einen Computer mit Hilfe der Stimme zu steuern, gibt es in den Bereichen der Steuerung durch Muskel- und Gehirnaktivität die meisten Potentiale. Im Vergleich sind bei den anderen beschriebenen Eingabesystemen die Fragen von zentraler Bedeutung, wie gewisse Fehler ausgebessert (z.B. Lichtprobleme bei der Augensteuerung) bzw. die Leistungsfähigkeit der Systeme gesteigert werden können. Hingegen bei der Steuerung durch Muskel- und Gehirnaktivität liegt der Fokus auf der Herstellung von marktreifen Geräten, die im Alltag eingesetzt werden können. Das Unternehmen Facebook%
\footnote{www.facebook.com}
%
arbeitet beispielsweise an einem System, dass Benutzern ermöglichen soll 100 Wörter pro Minuten durch reine Gedankensteuerung tippen zu können \cite{Facebook}.
\newline \newline
Die Frage, die man sich in Zukunft, vor allem bei der Steuerung durch Gehirnaktivität, stellen muss, ist, wie die rechtlichen Rahmenbedienungen ausschauen. Gibt es ausreichend großes Zielpublikum, dass sich von einer Software die Gedanken lesen lässt. Wer entscheidet welche Informationen für das Generieren eines Textes relevant sind und wer besitzt aller Zugang zu diesen Daten? Sollten diese Fragen ausreichend und transparent geklärt werden, gibt es in diesem Bereich das meiste Zukunftspotenzial.



