\chapter{Fazit und Ausblick}

Alternative Interaktionsmöglichkeiten ohne die Verwendung der Hände gewinnen stets an Bedeutung. Die verschiedenen Systeme finden immer mehr Einsatz, sowohl im Alltag, als assistierende Technologie oder im Arbeitsbereich.
\newline \newline
In letzter Zeit gewannen Spracheingabesysteme immer mehr an Beliebtheit, wie an dem Beispiel von Amazon Echo zu sehen ist. Weiters gibt es bereits auch einen Sprachassistenten, der von Google entwickelt wurde. Der Google Assistent%
\footnote{https://assistant.google.com/}
%
hat die selben Ziele, wie das System von Amazon - Fragen zu beantworten, Aufgaben zu übernehmen und technische Geräte im Haus zu steuern. Laut Futurezone \cite{Futurezone} wird auch das Unternehmen Apple im nächstes Jahr einen Lautsprecher mit digitaler Assistenz herausbringen. Das Grundkonzept baut auf dem bereits existierenden Sprachassistenten Siri auf.
\newline \newline
Obwohl es schon viele verschiedene alternative Systeme für die Interkation mit einem Computer gibt, gibt es dennoch Verbesserungspotential. Bei den meisten Eingabesystemen ist die Fragen von zentraler Bedeutung, wie gewisse Fehler ausgebessert (\zB Lichtprobleme bei der Augensteuerung) bzw. die Leistungsfähigkeit der Systeme gesteigert werden können. Hingegen liegt bei der Steuerung durch Muskel- und Gehirnaktivität der Fokus auf der Herstellung von mehr marktreifen Geräten und wie die Signale bei nicht invasiven Systemen verstärkt werden können. Das Unternehmen Facebook%
\footnote{www.facebook.com}
%
arbeitet beispielsweise an einem System, dass Benutzern ermöglichen soll 100 Wörter pro Minuten durch reine Gedankensteuerung tippen zu können \cite{Facebook}.
\newline \newline
Die Frage, die man sich in Zukunft, stellen muss, ist, wie die rechtlichen Rahmenbedienungen ausschauen. Gibt es ein ausreichend großes Zielpublikum, dass sich von einer Software die beispielsweise die Gedanken lesen lässt. Wer entscheidet welche Informationen für das Generieren eines Textes relevant sind und wer besitzt aller Zugang zu diesen Daten? Welche Daten werden bei einem Sprachassistenten, der permanent zuhört, gespeichert bzw. wie werden sie verwendet? Um eine größere Akzeptanz bzw. ein größeres Zielpublikum erreichen zu können, müssen diese Fragen ausreichend und transparent geklärt werden.



