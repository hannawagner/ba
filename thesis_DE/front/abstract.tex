\chapter{Abstract}

\begin{english} %switch to English language rules
%
This thesis provides an insight into various alternative input and \mbox{output methodes,} \mbox{compares} all the individual methods with each other and describes their application areas.
\newline \newline
There are many areas of application where an interaction with a system using the hands is not possible. Alternatives are necessary in daily situations, for example while driving, during strenuous and precise work on machines or for people with impairments.
\newline \newline
There are various possibilities how users can interact with a system without using their hands. There is the option to interact using voice control, eye control, gesture control, muscle control or even control based on brain activity. Some systems need to be sufficiently trained and calibrated for a successful interaction, others can be used \mbox{immediately}. There is not only a time difference for the preparation, but also the \mbox{interaction} itself differentiates. Some systems are using a joystick-like element as a \mbox{basis} for the interaction, some a camera and others are using sensors. 
\newline \newline
A computer can give feedback either through audio, haptic or visual output. \mbox{Various} possibilities are used depending on the area of application.
\newline \newline
Alternative interaction options are used in everyday situations, at work and as \mbox{assistive} technologies. Due to the current popularity of Amazon Alexa and other language \mbox{assistants}, the language output is defined as the most popular type of alternative \mbox{interaction} systems at the moment.
%
\end{english}


