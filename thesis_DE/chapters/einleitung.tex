\chapter{Einleitung}
\label{cha:Einleitung}

\section{Motivation und Zielsetzung}
%
Die meisten Interaktionen mit elektronischen Geräten jeder Art finden durch den Einsatz unserer Hände statt. Das reicht von der einfachen Bedienung von Geräten (z.B. Einschalten eines Computers) bis hin zur Steuerung von Abläufen mittels Gestik (z.B. Liederwechsel durch eine Wischgestik, die von einer Software erkannt und in einen Befehl umgewandelt wird). Doch nicht immer gibt es die Möglichkeit mit Hilfe der Hände zu interagieren. Beispielsweise sollen beim Autofahren beide Hände am Lenkrad bleiben, bei anstrengenden und präzisen Arbeiten an Maschinen müssen oft beide Hände benutzt werden oder aber auch für Menschen mit Tetraplegie bzw. Tetraparese, die ihre Hände und Arme nicht oder oft nur sehr eingeschränkt benutzen können, sind alternative Interaktionsmöglichkeiten äußerst nützlich.
\newline \newline
Ziel der Arbeit ist es einen umfassenden Überblick mit samt der Vor- und Nachteile, sowie Einsatzgebiete über alternative Eingabe- und Ausgabemethoden zu geben.

\section{Aufbau}
%Kapitel 2
Kapitel ~\ref{cha:Eingabe} stellt die einzelnen Eingabemethoden vor. Diese werden einerseits aus einer technischen Sicht (wie funktioniert das System), andererseits aus der Interaktionssicht (was muss ein Benutzer für eine erfolgreiche Interaktion machen bzw. beachten) beschrieben.
\newline \newline
%Kapitel 3
Anschließend werden in Kapitel ~\ref{cha:Ausgabe} Ausgabemethoden vorgestellt. Diese Ausgabemethoden müssen nicht unbedingt an eine bestimmte Eingabemethode gebunden sein. Das bedeutet, dass beispielsweise Systeme, die mit Hilfe der Augen gesteuert werden, auch ein akustisch Feedback geben können. 
%Kapitel 4
\newline \newline
Abschließend werden in Kapitel ~\ref{cha:Vergleich} die Vor- und Nachteile der einzelnen alternativen Eingabemöglichkeiten erläutert, gesonderte Rahmenbedingungen vorgestellt und Einsatzgebiete zu den einzelnen Systemen beschrieben.





